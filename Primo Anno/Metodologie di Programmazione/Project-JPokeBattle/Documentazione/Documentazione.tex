\documentclass[12pt]{article}
\usepackage{graphicx} % Required for inserting images
\usepackage{blindtext}
\usepackage{titlesec}
\usepackage{relsize}
\usepackage{fancyhdr}
\usepackage{listings}
\usepackage[italian]{babel}
\usepackage{xcolor}
\usepackage{array}
\usepackage[left=3cm,right=3cm, margin=3cm, vmargin=3cm]{geometry} % Imposta i margini laterali a 3 cm
\usepackage[utf8]{inputenc}
\usepackage{tcolorbox} % pacchetto per creare scatole colorate
\definecolor{codegreen}{rgb}{0,0.6,0}
\definecolor{codegray}{rgb}{0.5,0.5,0.5}
\definecolor{codepurple}{rgb}{0.58,0,0.82}
\definecolor{backcolour}{rgb}{0.95,0.95,0.92}
\lstdefinestyle{mystyle}{
    backgroundcolor=\color{backcolour},   
    commentstyle=\color{codegreen},
    keywordstyle=\color{magenta},
    numberstyle=\tiny\color{codegray},
    stringstyle=\color{codepurple},
    basicstyle=\ttfamily\footnotesize,
    breakatwhitespace=false,         
    breaklines=true,                 
    captionpos=b,                    
    keepspaces=true,                 
    numbers=left,                    
    numbersep=5pt,                  
    showspaces=false,                
    showstringspaces=false,
    showtabs=false,                  
    tabsize=2
}
\lstset{style=mystyle}

\begin{document}
\begin{titlepage}
    \begin{center}

        \includegraphics[width=0.75\textwidth]{Uniroma1.svg.png}
        \par\medskip\noindent \\

        \large{Università "Sapienza" di Roma\\

            Facoltà di Informatica\\

            \textbf{Corso}: Metodologie Di Programmazione}\\

        \vfill

        \textbf {\Huge{Documentazione JPokeBattle}}
        \par\medskip\noindent \\
        \large{\textbf{Author: }Alessia Cassetta}\\

        \large{\textbf{Matricola: }2113909}


        \vfill

        Giugno 2024 % DATA

    \end{center}
\end{titlepage}
\newpage
\tableofcontents
\newpage

\section{Feature Sviluppate}
\begin{itemize}
    \item \textbf{Minimo}
          \begin{itemize}
              \item Implentati 55 pokemon di prima generazione, con le loro mosse base e statistiche.
              \item Assegnare a tutti i Pokémon le due mosse di tipo neutro apprese al livello 1.
              \item Implementare le schermate “start”, battaglia, cambio pokémon, “you win”, e “game over”.
              \item Adottata Java Swing
              \item Far affrontare al giocatore una serie di avversari NPC, fino alla sua prima sconfitta.
          \end{itemize}
    \item \textbf{Tipico}
          \begin{itemize}
              \item Preservare lo stato dei pokémon del giocatore nella serie di lotte.
              \item Implementate tutte le mosse dei Pokémon scelti, rispettando le loro meccaniche di funzionamento dipendenti dai loro tipi, ma ignorando i cambiamenti di stato come avvelenamento, stordimento, etc.
              \item Implementare una schermata leaderboard che mantenga i 10 record migliori.
          \end{itemize}
    \item \textbf{Extra}
          \begin{itemize}
              \item Set crescita:
                    \begin{itemize}
                        \item Implementare punti individuali e punti allenamento che migliorino le capacità dei pokémon sulla base delle vittorie, aggregandoli appropriatamente.
                        \item Implementare i meccanismi di passaggio di livello ed evoluzione dei Pokémon, incluso l’apprendimento di nuove mosse a determinati livelli.
                    \end{itemize}
              \item Set battaglia:
                    \begin{itemize}
                        \item Implementare strategie per un comportamento “intelligente” degli avversari NPC, per supportare un’esperienza di gioco appagante.
                    \end{itemize}
          \end{itemize}
\end{itemize}
Queste Feature sono state implementate in modo incrementale, partendo dal Minimo e aggiungendo le funzionalità Tipiche e Extra. Ogni Feature è stata testata singolarmente, per garantire il corretto funzionamento del codice.


\newpage
\section{Decisioni di progettazione}
Il progetto è basato su un'architettura a classi, in cui ogni classe rappresenta un'entità del gioco. Le classi principali sono:
\begin{itemize}
    \item \textbf{\texttt{\textcolor{red}{Pokemon}}}: rappresenta un Pokémon, con le sue statistiche, mosse e punti esperienza.
    \item \textbf{\texttt{\textcolor{red}{Ability}}}: rappresenta una mossa, con il suo nome, tipo, potenza, precisione e punti mossa.
    \item \textbf{\texttt{\textcolor{red}{Coach}}}: rappresenta un allenatore, con il suo nome, i suoi Pokémon e il suo stato.
    \item \textbf{\texttt{\textcolor{red}{Battle Frame}}}: rappresenta una schermata, con i suoi elementi grafici e le sue azioni.
    \item \textbf{\texttt{\textcolor{red}{Battle}}}: rappresenta una battaglia, con i suoi stati e le sue azioni.
    \item \textbf{\texttt{\textcolor{red}{Change Pokemon}}}: rappresenta un cambio Pokémon, con i suoi stati e le sue azioni.
\end{itemize}
Ogni classe è stata progettata per essere il più possibile coesa e con un'alta coesione. Inoltre, è stata adottata l'ereditarietà per le classi \texttt{\textcolor{red}{Pokemon}} e \texttt{\textcolor{red}{Ability}}, in modo da poter creare facilmente nuovi Pokémon e nuove Mosse. 
\\ \\
La schermata più importante è la schermata della \textbf{battaglia}, in cui il giocatore può scegliere la mossa da usare o cambiare Pokémon. Questa schermata viene implementata da \texttt{\textcolor{red}{BattleFrame}} che estende la classe JFrame, e contiene tutti gli elementi grafici e le azioni che si possono fare durante la battaglia.
Essa implementa l'interfaccia \texttt{\textcolor{red}{BattleEventListener}} che contiene i metodi per gestire le azioni del giocatore durante la battaglia. \texttt{\textcolor{red}{BattleEventListener}} comunica con la classe \texttt{\textcolor{red}{Battle}}, che contiene la logica della battaglia e i suoi stati.
\\ \\


\end{document}